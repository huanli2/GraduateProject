%%==========================
%% chapter01.tex for SJTU Master Thesis
%% based on CASthesis
%% modified by wei.jianwen@gmail.com
%% version: 0.3a
%% Encoding: UTF-8
%% last update: Dec 5th, 2010
%%==================================================

%\bibliographystyle{sjtu2} %[此处用于每章都生产参考文献]
\chapter{绪论}
\label{chap:background}

\section{研究背景}

在计算机图形学领域,计算机模拟的目标在于增强虚拟现实的真实感。烟雾、水、空气等流体行为是自然环境的重要组成部分,计算机模拟自然界中这些流体的行为,能很好地烘托环境的气氛。传统的动画开发需要由动画师自己手动绘制流体动画的每一帧,虽然这样的创作方式可以让动画师更加灵活地创作,但是同时也需要巨大的工作量,并且手工绘制的动画难以获得高精度和复杂变化的流体效果。随着流体动画技术和相关应用需求的发展,流体动画产业越来越多地采用计算机来模拟和渲染流体的行为和现象,成为计算机图形学领域的一个非常重要的研究分支,并且已经在影视游戏、虚拟现实和科学计算等领域取得了广泛的应用价值。在计算机图形学领域,近年来已经出现了一大批稳定、逼真、并且计算较为高效的流体模拟方法,已经大量地应用到了影视特效和游戏等产业。例如在电影设计的辅助下,现有的流体模拟方法~\cite{kim2008wavelet}~\cite{losasso2004simulating}~\cite{golas2012large}成功地模拟出了海浪奔涌、烟雾缭绕等震撼人心的效果。

但是,随着电影和游戏产业的迅速发展,对大规模场景下的复杂流体动画的模拟技术的要求也越来越高。而现有的流体模拟方法都受限于大规模场景的海量网格数据的计算,导致在计算较大规模的流体动画的每一帧时,需要消耗几十甚至几小时的时间。因此,研究如何快速高效地模拟大规模场景下的流体动画技术具有十分重要的意义。现有研究中,人们提出了很多方法改进流体模拟的计算速度,但是仍然存在很多问题。虽然有些方法通过简化流体力学模型能达到加速的目的,但其计算耗费仍然非常巨大。

压缩感知理论~\cite{donoho2006compressed}~\cite{candes2006near}~\cite{candes2008introduction}颠覆了经典的尼奎斯特(Nyquist)采样定理,在学术界被誉为一个“Big Idea”。该定理在2006年诞生后的短短几年里,就很快在通信、声音、图像等领域成功应用,并取得了令人难以置信的突破。基于压缩感知理论的过完备稀疏训练字典技术~\cite{yang2012coupled}指出,如果训练字典足够完备,就可以利用重构上采样方法准确地重构出原信号数据。近年来,过完备稀疏字典也在图像处理、信号处理、分类等领域得到了成功应用。本课题拟针对大规模流体动画的快速高效模拟问题,探索将压缩感知理论技术应用到流体动画领域的方法与框架。

\section{国内外研究现状}

本文研究的是一个交叉性的课题,将基于稀疏编码技术的过完备训练字典技术应用到流体模拟领域,达到提高流体动画模拟方法的计算速度的目的。在该部分章节中,我们将会重点介绍流体模拟方法的国内外现状,并简单描述基于稀疏编码技术的过完备训练字典方法的发展过程和要解决的问题。

\subsection{流体模拟的基本方法}
\label{sec:basicMethod}
 计算机动画领域中的流体动画模拟方法,不同于传统工程计算中的计算流体力学方法,追求高精度的数值计算,而是更加看重视觉效果和计算效率,故计算机动画中使用的流体动画模拟算法具有高效、简单和易于实现等特点。基于物理的流体动画方法通过求解Navier-Stokes物理方程组来模拟流体,主要有三种基本模拟方法:欧拉网格法、拉格朗日粒子法和基于旋度的方法。
 
 \subsubsection{欧拉网格法}
\label{sec:Euler}

欧拉网格法通过在空间中设置固定的点,观测流体的物理量(如速度场、压强场、温度场等)在通过该点时随着时间的变化来表示流体的运动。通常在欧拉方法中,流体空间被划分成规则或不规则的网格,通过在这些离散网格上求解Navier-Stokes方程组得到一定时间序列中这些离散网格上的物理量的变化过程,从而达到描述流体运动的目的。1997年Foster和Metaxas~\cite{foster1997modeling}建立一种气体在热气浮力作用下运动的气体运动模型,率先把流体模拟引入进算计图形学领域。1999年Stam~\cite{stam1999stable}首次定义了一种投影的方法,将流体模拟的基础模型分解为四个主要部分分别求解,同时还引入了一种无条件稳定的半拉格朗日方法,奠定了欧拉网格法的基础。Fedkiw等人~\cite{fedkiw2001visual}在2001年提出的对烟模拟的框架,基本完善了欧拉网格法的步骤和流程;Enright等人~\cite{enright2002animation}使用高精度网格模拟复杂流体表面,至此,基于欧拉方法的流体模拟框架基本趋于稳定。

 \subsubsection{拉格朗日粒子法}
\label{sec:Lagrangian}

拉格朗日粒子法将流体当做一个粒子系统来处理。模拟区域中的流体或者固体点都相应地标记成一个独立的、具有速度和位置属性的粒子。这些粒子在外力和周围其他粒子的相互影响下,物理量随着时间的变化而变化,拉格朗日粒子法使用这些变化的物理量描述流体的运动。M{\"u}ller,Adams~\cite{muller2003particle}~\cite{adams2007adaptively}等人先后采用拉格朗日方法快速高效地模拟出流体动画效果。相较于欧拉网格方法,拉格朗日粒子法计算量更小,具有模拟速度快的优点。但是该方法无法保证流体的无散条件,难以生成稳定的动画效果,并且渲染较为困难,不适合需要很高真实感的流体动画场合。

 \subsubsection{基于旋度的方法}
\label{sec:Vortex}

基于旋度的方法通过描述涡旋粒子随着时间的运动来模拟流体动画的效果。该类方法通过计算涡旋粒子的旋度变化来计算流体的速度变化,因流体的细节主要来自于涡旋粒子,故该方法可以生成较多的流体细节。在2005年和2006年,Alexis~\cite{angelidis2005simulation}~\cite{angelidis2006controllable}等人相继提出了基于旋度的流体模拟方法,并且成功地生成了烟雾缭绕的效果。基于旋度的方法具有计算效率高、可快速模拟流体动画的优点,但是该类方法难以处理各种边界条件,如流体的自由表面和固体边界等。另外,单独使用该种类型的方法,难以保证模拟的稳定性。

\subsection{流体模拟优化方法}
\label{sec:optimization}
相较于其他方法,欧拉网格法因易于离散、能够保证无散条件等优点,相关的研究工作最多。但是欧拉方法的模拟效果依赖于网格精度,想要模拟出涡旋等细节,需要在足够高的网格精度上模拟才能实现。但是提高网格的精度会大幅度地增加模拟的时间开销,常常会消耗几个甚至几十个小时来计算很小一段时间帧的流体动画。在学术界,拟针对如何提高流体模拟的计算效率,提出了各种优化方法,主要相关工作可归纳为四类。

\subsubsection{自适应网格方法}
\label{sec:adaptive-gird}

在流体模拟过程中,并不是所有的模拟区域都包含有丰富的细节特征。自适应网格方法针对流场不同区域的不同级别的细节特征,分配不同粗细的网格。对包含丰富流体细节的流场区域,分配较细的网格,而对变化平缓、流体细节较少的流场区域,则分配较粗的网格。如文献~\cite{losasso2004simulating}提出的八叉树方法,使用八叉树结构对流体区域进行自适应划分,在边界区域等包含丰富流体细节的区域使用精细的网格,在其他流体细节较少的区域使用粗略网格。该类方法的优点是能够在总体上使用较少网格数量的同时,模拟出流体动画的细节效果,达到提高流体模拟的计算效率的目的。但是自适应网格方法要控制不同级别的网格精度,增加了算法的数据处理和计算的复杂性。

\subsubsection{混合模型方法}
\label{sec:mix-model}

混合模型方法的基本思想是根据流场不同区域的计算要求,采用不同类型的计算模型。现有的该类方法有混合网格法~\cite{dobashi2008fast}、网格-粒子混合法~\cite{zhu2010creating}、欧拉-旋度混合法~\cite{golas2012large}等。由于这类优化方法结合了不同计算模型的优点,可在优化计算速度的同时,保留流体动画尽可能多的细节。如文献~\cite{dobashi2008fast}提出的方法混合使用不同的网格计算模型达到加速的目的;文献~\cite{zhu2010creating}使用粒子的方法求解对流步骤,但是在求解投影步骤时使用网格的方法保证流体的无散条件;文献~\cite{golas2012large}采用欧拉网格处理自由边界和障碍物边界等边界条件,有效地避开了基于旋度方法不能处理各种边界条件的缺陷,在其他无边界区域采用基于旋度的方法,又具有基于旋度的模拟方法的计算速度快的优点。但是这类方法需要维护不同计算模型数据之间的耦合,其算法的设计与实现比较复杂,并且数据的耦合计算过程也会引入额外的计算误差。

\subsubsection{低—高精度重采样方法}
\label{sec:low-high-resampling}

欧拉网格方法通常分成三个主要步骤:对流、添加外力和投影。当提高网格精度时,计算投影步骤所消耗的时间会以指数级的速度增长,成为该方法所有步骤中耗时最为显著的部分。低—高精度重采样方法的基本思想就是尽量减少投影步骤的计算时间,其具体做法为将高耗时的投影步骤放到低精度网格计算。Yoon等人~\cite{yoon2009procedural}和Kim等人~\cite{kim2008wavelet}先在低精度网格上计算流体场,再用双三次插值的方法将其上采样到对应的高精度网格上,使用涡旋粒子或者小波湍流合成高精度流场细节。M.Lentine ~\cite{lentine2010novel}在高精度网格上计算流场数据,然后再将高精度网格流场数据降采样到对应的低精度网格上,在低精度网格中求解泊松方程,保证低精度网格上流场数据的无散条件,最后将其上采样到高精度网格,并在每个粗网格内局部调整流体速度场数据,保证其在高精度网格中的无散条件。前人的工作表明,这种类型的优化模拟方法确实减少了泊松方程求解的时间开销,但是目前还缺乏针对流体场数据设计的计算高效和准确的重采样方法。

\subsubsection{模型缩减法}
\label{sec:ModelReduction}

模型缩减的方法的核心思想是对流体的数据和计算模型降维处理,然后在低维度空间中求解方程,达到提高流体模拟速度的目的。Treuille~\cite{treuille2006model}提出一种实时的模型缩减方法,先使用离线的高精度模拟器对一个给定场景做预计算,并求解生成一组能表示流场数据特征的正交基函数。然后在实时的流体模拟计算过程中,先将流体速度场映射到基函数空间,再求解流体方程。由于基函数空间的维度远小于原计算模型空间的维度,故可大幅提高流体模拟的计算效率。文献~\cite{wicke2009modular}将分块的思想引入到上述算法中,先对场景进行分块,再在每个分块的对场景中采用模型缩减的方法进行模拟。但是这种方法不能灵活变换场景,并且大幅降低了计算模型的准确度。

\subsection{过完备稀疏字典技术}
\label{sec:overcompleteDictionaries}

压缩感知理论是2006年由Donoho,Candes等数学家提出的一个数学理论~\cite{candes2006near}~\cite{donoho2006compressed}~\cite{candes2008introduction},该定理在诞生后的短短几年里,就很快在通信、声音、图像等领域成功应用,并取得了令人难以置信的突破。文献~\cite{kroeker2009face}结合稀疏编码技术和压缩感知理论,成功地识别出完全被噪声遮蔽,连人眼都无法辨认的图像。

近年来,基于压缩感知技术的过完备稀疏字典技术也成为了一个热门话题。过完备稀疏字典技术假设低-高精度信号之间的关系可以由其稀疏表示形式表示,而数据的稀疏表示模型也假设信号可以由一个预定义字典的很少量原子的线性组合表示,因此选择一个使信号稀疏化的字典是一个极其重要的问题。文献R. Rubinstein~\cite{rubinstein2010dictionaries}描述了选择字典的两种方法: (1)使用基于数据的数学模型建立一个稀疏字典;(2) 使用训练数据集学习一个训练字典。第一种方法基于谐波分析,对一个更简单的数学运算建模并根据该模型设计一个高效表示。如傅里叶字典根据平滑函数建模,小波字典根据含奇异点的分段平滑函数建模。该类字典具有不涉及字典矩阵乘法、快速隐实现的特点,但是该类字典使用的数学模型过于简单,不能很好地表示自然现象的复杂性。第二种方法假设复杂自然现象的结构可以从数据本身更直接地提取,这种直接提取的好处是可以更好地适应特定数据。\(l^0\)范数指矩阵中非0元素的个数,矩阵中非0元素越少,说明矩阵越稀疏,故\(l^0\)范数直观上可以表示矩阵的稀疏度。\(l^1\)范数指矩阵中各元素绝对值之和,是\(l^0\)范数的最优凸近似,故\(l^1\)范数也可以实现稀疏。近年来,基于训练数据集的字典方法受到稀疏表示理论影响,都集中于研究\(l^0\)和\(l^1\)范数,使得基于训练数据集的字典方法可以应用最新的稀疏编码技术提取数据的特征,实现得以简化。

过完备稀疏字典技术融合基于训练数据集的训练字典和基于数学模型的分析字典的优点,使用稀疏编码技术学习特定的训练数据,建立一个可以使训练数据足够稀疏化的字典。由于该字典拥有比原数据信号更多维度的基信号原子,故我们称该字典是过完备的。过完备字典技术指出,一个信号在其表示域可以表述成多种形式,能否选取最合适的表示取决于使用的字典和解码方法。1993年,文献~\cite{mallat1993matching}提出了一种基于冗余函数字典的新奇的分解方案;两年后,Chen, Donoho和Saunders发表了一篇基于基追踪(Basis Pursuit)的分解算法~\cite{chen1998atomic},他们的工作奠定了使用字典的稀疏信号表示方法的基础。多年来,正交和双正交的字典表示方法因其简单性和有效性在信号处理领域被广泛应用,如小波去噪~\cite{donoho1995noising}.但是该类型的字典方法受限于其表征能力,于是产生了过完备字典技术的发展。

应用过完备训练字典的重构上采样方法以低精度信号为输入,根据离线生成的过完备训练字典分解计算出对应的稀疏表示,从而推断出原高精度空间信号。该类方法的重构效果对训练字典的依赖性很大,训练字典越完备,其重构出的结果越准确;训练字典对训练数据的依赖性也很大,使用的训练数据集越丰富,生成的字典也越完备。Yang等人~\cite{yang2010image}~\cite{yang2008image}~\cite{yang2012coupled}将压缩感知技术应用到图像重构领域,达到了令人瞩目的效果;文献~\cite{marwah2013compressive}采用类似的方法,将压缩感知技术应用到照相机的光场重构,也取得了较为理想的效果。近年来,基于稀疏编码技术的过完备训练字典技术和重构上采样技术也成为了各领域研究的一个热门话题,但是在流体动画领域,还没有任何使用感所感知技术和过完备训练字典技术的相关文献。

\section{本文的主要工作及创新点}
\label{sec:creation}

\subsection{主要研究工作}

本文借鉴流体模拟优化方法中的低-高精度重采样方法,将流体模拟中最为耗时的投影步骤放到低精度网格模拟,提高投影步骤求解泊松方程的计算效率。然后使用基于稀疏编码的过完备稀疏训练字典技术,将低精度网格流体速度场数据重构上采样到高精度网格空间,用于下一帧模拟计算,达到在尽量少的时间耗费下提供尽可能多的高精度流体细节的目的。

本文的主要研究工作包括如下部分:
\begin{itemize}
\item 通过学习复杂流体数据的稀疏表示和疏密样本数据之间的映射关系,生成适用于流体速度场的过完备训练字典。实际上,自然界的流体现象可以看作是由许多局部细微结构组成的,因此可以尝试将复杂流体数据用局部细微结构的线性组合来表示,这些局部细微结构可以从大量流体数据中采集、提取和存储,形成一个经验数据集。如果这样的一个经验数据集可以表示几乎所有的流体现象的局部细微结构,我们称其为过完备的训练字典;
\item 研究流体疏密数据重构上采样方法。利用压缩感知理论中的稀疏逼近手段,探索一个适合复杂流体疏密速度场数据的重构上采样方法;
\item 建立2D场景流体动画压缩重采样的高效计算框架。借鉴已有的低—高精度重采样方法,在高精度流场上计算对流和加外力步骤,然后将高精度流场数据降采样投影到低精度网格计算投影步骤,保证无散条件,再使用过完备稀疏训练字典,将低精度流场数据重构上采样到高精度网格的流体场中;
\item 解决应用过完备稀疏字典的重构上采样方法到流体动画模拟中导致的流体形态改变、边界条件等问题。
\end{itemize}

\subsection{本文创新点}

目前,基于物理的计算机流体模拟技术已经较为成熟,但是如何快速高效地模拟流体动画而尽可能多且真实地保留住流体动画的细节,一直是一个研究的热点问题。本课题拟针对这样一个问题,探索稀疏编码理论在流体动画领域的适用性,对流体动画领域的研究有重要的意义。

本课题的研究内容十分新颖,主要包括如下创新点:
\begin{itemize}
\item 将基于稀疏编码技术的过完备稀疏训练字典技术引入到流体动画研究领域,学习复杂流体的局部细微结构,探索应用过完备训练字典的重构上采样技术在流体动画领域的可行性与适用性;
\item 通过降采样矩阵与高精度稀疏字典的线性组合,计算生成适用于低精度流体速度场空间的稀疏字典,并根据该低精度空间的稀疏训练字典获得低精度速度场的稀疏表示形式;
\item 基于稀疏编码技术,系统地探索复杂流体数据的稀疏表示形式,建立基于局部细微结构的稀疏表示形式,提出适用于流体动画的重构上采样方法与框架。
\end{itemize}

\section{本文组织结构}
\label{sec:orgnization}

本文共有六个章节,第一章绪论部分主要介绍了本课题的研究背景与国内外现状,后面的五个章节依次为:

第二章主要介绍流体动画物理模型Navier-Stokes方程组的推导过程及其在计算机图形学领域的基本求解方法,并在第二章的后面部分简要介绍基于稀疏编码技术的过完备训练字典技术要求解的问题。

第三章将会给出本文提出的基于稀疏过完备训练字典技术的流体重构上采样框架,并详细介绍该框架中的一些基本步骤的求解方法。

第四章主要介绍本文的重构上采样方法的整体框架,并详细描述本文提出的应用降采样矩阵的过完备训练字典方法及其重构上采样方法的具体实现过程。

在第五章中,我们将会给出大量的实验结果,探讨本文提出方法的可行性。首先,我们会在图像域验证本文提出的重构算法,确定其与双三次插值方法相比,能够重构出更多的细节效果之后,再将本文的重构算法应用到流体模拟领域,并与模拟器直接模拟生成的低-高精度流体动画、双三次插值重构上采样生成的流体动画比较,证明本文提出的算法确实能在一定程度上增加流体的细节。在实验章节的最后部分,我们还会给出本文提出的重构方法与框架的时间开销。

最后一章将会对本文的工作做一个总结,分析本文方法的局限性以及未来可能的研究方向。

