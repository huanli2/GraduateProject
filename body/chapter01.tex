%%==========================
%% chapter01.tex for SJTU Master Thesis
%% based on CASthesis
%% modified by wei.jianwen@gmail.com
%% version: 0.3a
%% Encoding: UTF-8
%% last update: Dec 5th, 2010
%%==================================================

%\bibliographystyle{sjtu2} %[此处用于每章都生产参考文献]
\chapter{绪论}
\label{chap:background}

\section{研究背景}

烟雾、流水等流体是自然界环境的重要组成部分,计算机模拟的流体行为能增添虚拟现实的真实感,更好地烘托环境气氛。在传统的动画开发环境下,流体动画的每一帧为动画师手工绘制,虽然这样能灵活可控地进行创作,但所需工作量巨大,而且难以获得高精度效果。随着流体动画技术和相关应用需求的发展,流体动画采用计算机模拟和渲染流体现象,是 计算机图形学领域的一个非常重要的研究分支,在影视游戏、虚拟现实、科学计算等领域有广泛的应用价值。

在计算机图形学领域,近年来出现了一大批稳定、逼真、比较高效的流体模拟方法,并且已经大量地应用到了影视游戏等产业,在电影设计的辅助下,成功地模拟出了海浪奔涌、水珠迸溅、烟雾缭绕等震撼人心的效果。

但是,随着虚拟现实和数字媒体应用的发展,具有复杂流体动态效果的大规模流体动画将扮演日益重要的角色,而现有的流体模拟方法都受限于对大规模场景的海量动画数据的采样和计算,导致在计算较大规模的流体动画的每一帧时,需要消耗几十甚至几小时的时间。因此,研究大规模流体动画的快速高效模拟方法具有重要的应用意义。现有研究中,人们提出了很多方法改进流体模拟的速度,但是仍然存在很多问题。有些方法通过简化流体力学模型达到加速的目的,但其计算耗费仍然非常巨大。

压缩感知颠覆了经典的Nyquist定理,被誉为学术界的一个“Big Idea“。该定理在2006年诞生后的短短几年里,就很快在通信、声音、图像等领域成功应用,并取得了令人难以置信的突破。基于压缩感知理论的超完备稀疏字典技术指出,如果训练字典足够完备,就可以利用重构上采样方法准确地重构出原信号数据。近年来,超完备稀疏字典也已在图形处理、信号处理、分类等领域得到了成功应用。本课题拟针对大规模流体动画的快速高效模拟问题,探索将压缩感知理论技术应用到流体动画领域的方法与框架。

\section{国内外研究现状}
\subsection{流体模拟的基本方法}
\label{sec:basicMethod}
 计算机动画领域中的流体动画模拟方法,不同于传统工程计算中的计算流体力学追求高精度的数值计算,而是更加看重视觉效果和计算效率,故计算机动画中使用的流体动画模拟算法具有高效、简单和易于实现等特点。基于物理的流体动画方法通过求解Navier-Stokes物理方程组来模拟流体,主要有三种基本模拟方法:拉格朗日粒子法,欧拉网格法和基于旋度的方法。
 
 \subsubsection{欧拉方法}
\label{sec:Euler}

欧拉法通过在空间中设置固定的点,观测流体的物理量(如速度、密度、温度等)在通过该点时随着时间的变化来表示流体的运动。通常在欧拉方法中,流体空间被划分成规则或不规则的网格,通过在这些离散网格上求解Navier-Stokes方程组得到一定时间序列中这些离散网格上的物理量的变化过程,从而达到描述流体运动的目的。1997年Foster和Metaxas~\cite{foster1997modeling}建立一种气体在热气浮力作用下运动的气体运动模型,率先把流体模拟引入进算计图形学领域。1999年stam~\cite{stam1999stable}首次定义了一种投影的方法,将流体模拟的基础模型分解为四个主要部分分别求解,同时还引入了一种无条件稳定的半拉格朗日方法,奠定了欧拉法的基础。Fedkiw等人~\cite{fedkiw2001visual}在2001年提出的对烟模拟的框架,基本完善了欧拉法的步骤和流程;Enright等人~\cite{enright2002animation}在2002年提出了对复杂流体表面模拟的框架,至此,采用欧拉模拟流体的框架也基本稳定。

 \subsubsection{拉格朗日粒子法}
\label{sec:Lagrangian}

拉格朗日粒子法将流体当做一个粒子系统来处理。模拟区域中的流体或者固体点都相应地标记成一个独立具有速度和位置属性的粒子,通过描述这些粒子在受到外力和周围其他粒子互相影响之后在一定时间序列内的物理量变化来描述流体运动。M{\"u}ller,Adams~\cite{muller2003particle}~\cite{adams2007adaptively}等人采用了该方法模拟流体。拉格朗日方法具有计算量小,模拟速度快的优点,但是难以保持无散条件,并且难以从大量离散的粒子中提取出光滑液体表面用于渲染,不适合需要照片级真实感的场合。

 \subsubsection{基于旋度的方法}
\label{sec:Vortex}

基于旋度的方法通过描述涡旋粒子随着时间的的运动来模拟流体动画的效果。该类方法通过计算涡旋粒子的旋度变化来计算流体的速度变化,而流体的细节主要来自于涡旋粒子,故该方法可以生成较多的细节。但是难以处理各种边界条件,如自由表面和固体边界等,且难以保证模拟的稳定性。Alexis~\cite{angelidis2005simulation}~\cite{angelidis2006controllable}等人分别在2005年和2006年提出了基于Vortex Filament的基于旋度的流体模拟方法,并且成功地生成了烟雾缭绕的效果。

\subsection{流体模拟优化方法}
\label{sec:optimization}
相较于其他方法,欧拉法因易于离散和求解无散条件等优点,相关的研究工作最多。但是欧拉方法的模拟效果依赖于网格精度,想要模拟出涡旋等细节,需要使用足够高的网格精度才能实现。但是提高网格的精度会大幅度地增加模拟的时间开销,常常会消耗几个甚至几十个小时来计算很小一段时间帧的流体动画。在学术界,拟针对如果提高流体模拟的速度,提出了各种优化方法,主要相关工作可归纳为四类。

\subsubsection{自适应网格方法}
\label{sec:adaptive-gird}

在流体模拟过程中,并不是所有的模拟区域都包含有丰富的细节特征。自适应采样方法针对流场的不同级别的细节特征,分配不同粗细的网格,对细节丰富的高频流场区域,分配最细的网格,而对变化平缓的低频流场区域,则分配较细的网格。如八叉树方法~\cite{losasso2004simulating},对空间进行自适应的八叉树划分,在边界等重要细节区域划分出更精细网格,而将非重要区域划分为粗略网格。该类方法的优点是能在保持细节质量的同时,降低总体采样数量。但是由于要控制不同级别的网格精度,也使得算法的数据处理计算更为复杂。

\subsubsection{混合模型方法}
\label{sec:mix-model}

该类方法的基本思想是根据流场不同区域的计算要求,分别采用不同类型的计算模型。现有的该类方法有混合网格法~\cite{dobashi2008fast}、网格-粒子混合法~\cite{zhu2010creating}、欧拉-旋度混合法~\cite{golas2012large}等等。这类方法结合了不同模型的优点,可在优化计算速度的同时,保留流体动画尽可能多的细节。如文献~\cite{dobashi2008fast}全局采用粗略的网格,局部采用细致的网格以捕捉细节;文献~\cite{zhu2010creating}利用粒子进行对流,借助网格解压力项与无散条件;文献~\cite{golas2012large}采用欧拉网格处理自由边界和障碍物边界等边界条件,可以有效地避开基于旋度方法不能处理边界条件的缺陷,在其他无边界区域采用基于旋度的方法,充分地利用了基于旋度的方法模拟速度快的优点。但是这类方法需要维护不同不同计算模型及其之间的数据耦合,增加了算法设计和实现的难度,耦合过程中也会带来额外的精度损失。

\subsubsection{低-高精度重采样方法}
\label{sec:low-high-resampling}

欧拉方法通常分成三个步骤:对流、添加外力和投影。当提高网格精度时,计算投影步骤所消耗的时间会以指数级增长,成为该方法所有步骤中耗时最为显著的一个部分。低-高精度重采样方法的基本思想就是尽量减少投影步骤的计算时间,其具体做法为将高耗时的投影步骤放到低精度场计算。Yoon等人~\cite{yoon2009procedural}提和Kim等人~\cite{kim2008wavelet}提出一种过程化合成方法,先计算一个低精度流场,将低精度流场线性上采样,得到高精度流场,再使用涡粒子或者小波湍流合成高精度流场细节。M.Lentine ~\cite{lentine2010novel}等人采用高精度网格对液体表面追踪,通过液体无偏差滤波器将获得的高精度表面降采样到低精度网格,求解低精度下的泊松方程,保证无散条件,再将低精度的速度场映射回高精度网格,然后在每个粗网格内进行一次局部高精度网格的泊松方程求解,以保证高精度下速度场的无散条件,整体提高了高精度网格上的泊松方程求解效率。但是这类方法目前还缺乏高效准确的重采样理论和方法支持。

\subsubsection{模型缩减法}
\label{sec:ModelReduction}

该方法通过对流体数据和计算模型进行大幅缩减,达到实时模拟的目的。Andrie~\cite{treuille2006model}提出的模型缩减方法,针对一个给定场景,先用离线的高精度模拟器,预计算一组高精度流体模拟的结果,利用其得到一组代表流场特征的正交基函数,将流体速度场缩减变换到一个小的基函数空间中,然后在缩减模型中求解方程。由于其缩减后的变量个数不依赖于空间精度,故可大幅提升运算效率。但是其代价是大幅降低了模型准确度,并且不能灵活变换场景。~\cite{wicke2009modular}利用模块化思想加以改进,对场景分块计算,每个分块分别采用模型缩减的方法进行模拟,但是这种方法只适合于像拼版游戏对场景做少量重组变化,场景普适性差。

\subsection{过完备稀疏字典技术}
\label{sec:overcompleteDictionaries}

压缩感知是2006年由Donoho,Candes和Tao等数学家建立起来的一个严格的数学理论~\cite{candes2006near}~\cite{donoho2006compressed}~\cite{candes2008introduction},该定理在诞生后的短短几年里,就很快在通信、声音、图像等领域成功应用,并取得了令人难以置信的突破。~\cite{kroeker2009face}结合稀疏编码技术和压缩感知理论,成功地识别出完全被噪声遮蔽,连人眼都无法辨认的图像。近年来,基于压缩感知技术的过完备字典技术也成为了一个热门话题,

数据的稀疏表示模型假设信号可以由一个预定义字典中的很少量原子的线性组合表示,因此选择一个使信号稀疏化的字典是一个极其重要的问 题。R. Rubinstein~\cite{rubinstein2010dictionaries}描述了选择字典的两种方法: (1)基于数据的数学模型建立一个稀疏字典;(2) 使用训练数据集学习一个训练字典。第一种方法基于谐波分析,对一个更简单的数学运算建模,并根 据该模型设计一个高效表示。如傅里叶字典根据平滑函数建模,小波字典根据含奇异点的分段平滑函 数建模。该类字典具有不涉及字典矩阵乘法,快速隐实现的特点,但是相较于自然现象的复杂性,该类模型过于简单化。第二种方法假设复杂自然现象的结构可以从数据本身更直接地提取,这种直接提取的好处是可以更好地适应特定数据。近年来基于训练数据集的字典方法受到稀疏表示理论影响,都集中于研究\(l^0\)和\(l^1\)稀疏,使得方法得以简化并应用最新的稀疏编码技术。

疏密重构上采样方法,通过求解Lasso问题,得到输入信号的稀疏表示,然后将该信号的稀疏表示与训练字典线性组合,推断出输入信号的重构上采样信号。该方法的重构效果与方法依赖于过完备训练字典,不同的训练字典,需采用不同的重构上采样方法。Yang等人~\cite{yang2010image}~\cite{yang2008image}~\cite{yang2012coupled}将压缩感知技术应用到图像重构领域,达到了令人瞩目的效果;~\cite{marwah2013compressive}采用类似的方法,将压缩感知技术应用到照相机的光场重构,也取得了较为理想的效果。目前,在流体动画领域还没有任何使用感所感知技术的相关文献。

过完备稀疏字典技术融合训练字典和分析字典的优点,利用细数编码技术学习特定的训练数据,建立一个可以使训练数据足够稀疏化的字典。由于该字典拥有比原数据信号更多维度的基信号原子,故我们称该字典使过完备的。过完备字典技术指出,一个信号在其表示域可以表述成多种形式,能否选取最合适的表示取决于使用的字典和解码方法。1993年,~\cite{mallat1993matching}提出了一种基于冗余函数字典的新奇的分解方案;两年后,Chen, Donoho和Saunders发表了一篇基于basis pursuit分解的论文~\cite{chen1998atomic},他们的工作奠定了使用字典的稀疏信号表示方法的基础。多年来,正交和双正交的字典表示方法因其简单性和有效性在信号处理领域被广泛应用,如小波去噪~\cite{donoho1995noising}.但是该类型的字典方法受限于其表征能力,于是产生了过完备字典技术的发展。

\section{本文的主要工作及创新点}
\label{sec:creation}

实际上,自然界的流体现象可以看作是由很多局部细微结构组成的,因此,可以尝试将复杂流体数据用局部细微结构的线性组合来表示,这些局部细微结构可以从大量流体数据中采集、提取和存储,形成一个经验数据集,然后利用压缩感知理论中的稀疏逼近手段,将各种复杂流体数据转化为流体局部细微结构的稀疏表示。目前,流体领域还没有相关方法的文献出现,但是,在图像、音频与视频信号处理等领域,已经出现了大批相关的技术。本文借鉴流体模拟的过程化合成方法,将流体模拟中最为耗时的投影步骤放到低精度网格模拟,在较少的计算耗费下提供较多的流体模拟细节。然后使用基于稀疏编码的超完备字典技术,将低精度网格流场数 据重构上采样到高精度网格空间,用于下一帧模拟计算。

\subsection{主要研究工作}

基于压缩感知理论,本文尝试将应用在图像信号处理领域中的基于稀疏表示的过完备训练字典技术应用到流体动画领域,达到快速高效地模拟大规模流体动画的目的。

本文的主要研究工作包括如下部分:
\begin{itemize}
\item 学习复杂流体数据的稀疏表示和疏密样本映射,生成过完备的训练字典。实际上,自然界的流体现象可以看作是由许多局部细微结构组成的,因此可以尝试将复杂流体数据用局部细微结构的线性组合来表示,这些局部细微结构可以从大量流体数据中采集、提取和存储,形成一个经验数据集。如果这样的一个经验数据集可以表示几乎所有的流体现象的局部细微结构,我们称其为过完备的训练字典;
\item 流体疏密数据映射重构方法。利用压缩感知理论中的稀疏逼近手段,探索一个适合复杂流体疏密数据的重构方法;
\item 2D场景流体动画压缩采样的高效计算框架。借鉴已有的低-高精度重采样方法,在高精度流场上计算对流和加外力步骤,然后将高精度流场数据降采样投影到低精度网格计算投影步骤,保证无散条件,再使用过完备字典,将低精度流场数据重构到高精度流场中;
\item 解决疏密映射重构方法到流体模拟中导致的流体形态改变、边界条件问题。
\end{itemize}

\subsection{本文创新点}

目前,基于物理的计算机流体模拟技术已经较为成熟,但是如何快速地模拟流体动画而尽可能多且真实地保留住流体动画的细节,一直是一个研究的热点问题。本课题拟针对这样一个问题,探索压缩感知理论在流体动画领域的适用性,对流体动画领域有重要的意义。

本课题研究内容十分新颖,主要包括如下创新点:
\begin{itemize}
\item 将压缩感知理论引入流体动画研究领域,学习复杂流体细微结构,探索基于稀疏表示的过完备字典训练技术在流体动画领域的应用;
\item 通过线性组合高精度训练字典与一个下采样矩阵,计算生成能够表示低精度速度场空间的字典,并根据该低精度空间的字典获得低精度速度场的稀疏表示形式;
\item 基于压缩感知理论,系统地探索复杂流体数据的稀疏表示形式,建立基于局部细微结构的稀疏表示形式,提出适用于流体动画重构上采样的方法与框架。
\end{itemize}

\section{本文组织结构}
\label{sec:orgnization}

本文共有六个章节,第一章绪论部分主要介绍了本课题的研究背景与国内外现状,后面的五个章节依次为:

第二章主要介绍流体模拟的基础知识,主要包括流体模拟方法的一般方法和基本框架。该章节将会对基于网格的流体模拟方法做一个全面的介绍,并详细阐述流体模拟的基本数学模型——Navier-Stokes方程组。

第三个章节将会介绍基于稀疏编码的过完备稀疏字典方法的基础知识,并给出基于稀疏字典技术的流体重构上采样框架。

第四章会详细介绍本文的重构上采样方法,以及基于稀疏编码技术的过完备字典的训练方法。

在第五章节中,我们将会给出大量的实验结果,探讨本文提出方法的可行性。我们会先给出本文提出方法在图像域的实验结果,证明本文提出的重构算法在实际应用中确实能够重构出额外的细节之后,再将本文的重构算法应用到流体模拟领域,并与双三次插值的重构结果相比较,证明本文提出的算法确实能在一定程度上增加流体的细节。

最后一个章节将会对本文的工作做一个总结,分析本文方法的局限性以及未来可能的研究方向。

