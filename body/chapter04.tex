%%==================================================
%% chapter03.tex for SJTU Master Thesis
%% Encoding: UTF-8
%%==================================================

\chapter{常见问题与故障排除}
\label{chap:faq}

\begin{figure}
  \centering
  \includegraphics[width=0.3\textwidth]{chap2/testpng}
  \hspace{1cm}
  \includegraphics[width=0.3\textwidth]{chap2/testjpg}
  \bicaption[fig:SRR]{这里将出现在插图索引中}{中文题图}{Fig}{English caption}
\end{figure}

\subsubsection*{如何获得帮助和反馈意见}
你可以通过如下的途径反馈模板使用过程中遇到的问题:\href{https://github.com/weijianwen/sjtu-thesis-template-latex/issues}{开issue}
、\href{https://bbs.sjtu.edu.cn/bbsdoc?board=TeX_LaTeX}{水源LaTeX版}发帖,或者是给\href{mailto:weijianwen@gmail.com}{我}发送邮件---你可能需要好几天才能收到我的邮件回复。

\begin{table}[!hpb]
  \centering
  \bicaption[tab:firstone]{指向一个表格的表目录索引}{一个颇为标准的三线表格\footnotemark[2]}{Table}{A Table}
  \begin{tabular}{@{}llr@{}} \toprule
    \multicolumn{2}{c}{Item} \\ \cmidrule(r){1-2}
    Animal & Description & Price (\$)\\ \midrule
    Gnat & per gram & 13.65 \\
    & each & 0.01 \\
    Gnu & stuffed & 92.50 \\
    Emu & stuffed & 33.33 \\
    Armadillo & frozen & 8.99 \\ \bottomrule
  \end{tabular}
\end{table}
\footnotetext[2]{这个例子来自\href{http://www.ctan.org/tex-archive/macros/latex/contrib/booktabs/booktabs.pdf}{《Publication quality tables in LATEX》}(booktabs宏包的文档)。这也是一个在表格中使用脚注的例子,请留意与threeparttable实现的效果有何不同。}


下面一个是一个更复杂的表格,用threeparttable实现带有脚注的表格,如表\ref{tab:footnote}。

\begin{table}[!htpb]
  \bicaption[tab:footnote]{出现在表目录的标题}{一个带有脚注的表格的例子}{Table}{A Table with footnotes}
  \centering
  \begin{threeparttable}[b]
     \begin{tabular}{ccd{4}cccc}
      \toprule
      \multirow{2}{6mm}{total}&\multicolumn{2}{c}{20\tnote{1}} & \multicolumn{2}{c}{40} &  \multicolumn{2}{c}{60}\\
      \cmidrule(lr){2-3}\cmidrule(lr){4-5}\cmidrule(lr){6-7}
      &www & k & www & k & www & k \\
      \midrule
      &$\underset{(2.12)}{4.22}$ & 120.0140\tnote{2} & 333.15 & 0.0411 & 444.99 & 0.1387 \\
      &168.6123 & 10.86 & 255.37 & 0.0353 & 376.14 & 0.1058 \\
      &6.761    & 0.007 & 235.37 & 0.0267 & 348.66 & 0.1010 \\
      \bottomrule
    \end{tabular}
    \begin{tablenotes}
    \item [1] the first note.% or \item [a]
    \item [2] the second note.% or \item [b]
    \end{tablenotes}
  \end{threeparttable}
\end{table}

\section{参考文献管理}

参考文献的管理是这个学位论文模板又一个好玩的地方。

\begin{lstlisting}[caption={.bbl中被格式化之后的条目}, escapeinside="", numbers=none]
\bibitem["白云芬(2008)"]{bai2008}
  \textsc{"白云芬"}.
  \newblock {"信用风险传染模型和信用衍生品的定价"}[D].
  \newblock "上海: 上海交通大学, 2008."
\end{lstlisting}

再罗嗦两句,
.bst文件书写起来非常繁杂\footnote{可以参考\href{http://ftp.ctex.org/mirrors/CTAN/info/bibtex/tamethebeast/ttb_en.pdf}{《Tame The BeaST》}。},书写符合GBT7714标准的.bst文件更是一项浩大的工程。
因此,当大家为漂亮、标准的参考文献列表感到满意时,应该对GBT7714-2005NLang.bst的作者充满谢意。
作者在CTeX BBS发的帖子,请看
\href{http://bbs.ctex.org/viewthread.php?tid=33571&highlight=\%B2\%CE\%BF\%BC\%CE\%C4\%CF\%D7\%2BGB}{文后参考文献著录规则 GB/T 7714-2005}。
关于GB/T 7714-2005标准本身,请看\href{http://bbs.ctex.org/viewthread.php?tid=33571&highlight=GB\%2B\%B2\%CE\%BF\%BC\%CE\%C4\%CF\%D7}{这里}。

再多说两句,.bib是“参考文献的内容”,而控制参考文献表现(格式)的是.bst文件,本模板附带的是GBT7714-2005NLang.bst。

\subsection{在正文中引用参考文献}

\begin{lstlisting}[language={C}, caption={一段C源代码}]
#include <stdio.h>
#include <unistd.h>
#include <sys/types.h>
#include <sys/wait.h>

int main() {
  pid_t pid;

  switch ((pid = fork())) {
  case -1:
    printf("fork failed\n");
    break;
  case 0:
    /* child calls exec */
    execl("/bin/ls", "ls", "-l", (char*)0);
    printf("execl failed\n");
    break;
  default:
    /* parent uses wait to suspend execution until child finishes */
    wait((int*)0);
    printf("is completed\n");
    break;
  }

  return 0;
}
\end{lstlisting}

再给一个插入MATLAB代码的例子,感谢daisying站友提供的代码。

\begin{lstlisting}[language={matlab}, caption={一段MATLAB源代码}]
function paper1
r=0.05;
n=100;
T=1;
X=1;
v0=0.8;
sigma=sqrt(0.08);
deltat=T/n;
for i=1:n
    t(i)=i*deltat;
    w(i)=random('norm',0,t(i),1);
end
for i=1:n
    alpha(i)=0.39;
end
for i=1:n
    temp=0;
    for k=1:i
        temp=temp+alpha(k);
    end
    B(i)=exp(r*t(i));
    BB(i)=B(i)*exp(temp*deltat);
    BBB(i)=exp(-r*(T-t(i)));
end
for i=1:n
    s0(i)=X*BBB(i);
    v(i)=v0*exp((r-0.5*sigma^2)*t(i)+sigma*w(i));
    for j=i+1:n
        D=X*BBB(j);
        d1=(log(v(i)/D)+(r+sigma^2/2)*(t(j)-t(i)))/(sigma*sqrt(t(j)-t(i)));
        d2=d1-(sigma*sqrt(t(j)-t(i)));
        ppp(i,j)=D*exp(-r*(t(j)-t(i)))*(1-cdf('normal',d2,0,1))-v(i)*(1-cdf('n
ormal',d1,0,1));
    end
end
for i=1:n
    s1(i)=0;
    for j=i+1:n
        s1(i)=s1(i)+BB(j)^(-1)*alpha(j)*deltat*(X*BBB(j)-B(j)/B(i)*ppp(i,j));
    end
    s2(i)=0;
    for j=1:n
        s2(i)=s2(i)+alpha(j);
    end
    s2(i)=X*exp(-r*T-s2(i)*deltat);
    s(i)=BB(i)*(s1(i)+s2(i));
end
plot(s)
hold on;
plot(s0);
\end{lstlisting}

