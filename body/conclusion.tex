%%==================================================
%% conclusion.tex for SJTU Master Thesis
%% based on CASthesis
%% modified by wei.jianwen@gmail.com
%% version: 0.3a
%% Encoding: UTF-8
%% last update: Dec 5th, 2010
%%==================================================

\chapter{全文总结}%\markboth{全文总结}{}}
%\addcontentsline{toc}{chapter}{全文总结}

\section{工作总结}

在流体动画的基本模拟方法中,欧拉网格方法是最为常用的模拟算法,但是该算法的投影步骤是基本步骤中的瓶颈部分。目前,虽然出现了一大批较为成熟的求解投影步骤的方法,但是仍然不能解决精度提高时流体模拟的计算时间开销快速增长的问题,导致很难快速高效地模拟大规模流体动画。

基于稀疏编码技术,本文提出了一个低-高精度重采样的流体模拟的基本框架,该重构上采样框架首次将基于稀疏编码技术的重构上采样技术应用到流体动画领域中。为了探索出一对适合流体低-高精度速度场数据的训练字典方法和重构上采样方法,我们先后尝试了Yang提出的联合过完备训练字典方法、L1SR重构上采样方法和scSR重构上采样方法,但是这些现有的训练字典方法和重构上采样方法存在求解稀疏表示时,学习训练字典和重构上采样时的最优空间不一致,从理论上无法重构出准确的结果的问题,导致重构生成的流体动画也存在严重的形态问题。但是,通过L1SR和scSR重构上采样生成流体动画的实验,验证了本文提出的重构上采样框架的可行性。

针对L1SR和scSR重构上采样方法在实验中表现出的问题,我们从理论和实际的实验结果两个方面分析其原因,提出了本文的应用降采样矩阵的过完备训练字典方法。根据本文的实验结果可知,本文提出的重构上采样方法与流体动画重构上采样计算框架可以在一定程度上重构出类似于高精度网格模拟出的流体动画的细节。同时,该方法具有很强的通用性,并且易于移植。另外,如果以计算速度较慢的模拟器作为本文框架的基本模拟方法,本文方法能够在一定程度上提高模拟的速度;即使使用计算效率较高模拟器作为基本模拟方法,也可以通过增加流体局部 细微结构的大小,或者选择性地应用基于稀疏编码的过完备训练字典方法,达到提高流体计算速度的目的。

\section{存在的问题和未来工作展望}

在实验的过程中,我们发现本文提出的方法存在以下一些问题:

\begin{itemize}
\item 从实验结果可以看出,应用本文提出的基于稀疏编码的过完备训练字典的重构方法重构上采样速度场,能够重构出比双三次插值方法多很多的高精度流场细节。但是,本文提出的方法与框架模拟出的流体动画的形态与高精度流体动画的形态差别很大。因为重构算法不可能生成100\%的准确的重构结果,故目前这个问题无法解决。

\item 在我们的实验中,我们对速度场的水平分量和垂直分量单独做的重构上采样,并且训练字典也是单独训练生成的,故在我们的训练字典中,不能表示速度场的水平分量和垂直分量之间的关系,故我们不能保证重构结果的水平速度分量和垂直速度分量之间的关系是否正确。为了解决这个问题,在今后的研究中,可以考虑将速度场的水平分量和垂直分量放到一个向量中训练。另外,这样处理之后,可以将原来需要重构两次的计算开销放到一次计算中,故能减少一半的计算时间耗费。

\item 根据我们的实验结果,可以看出本文方法重构上采样生成的流体动画仍然存在形态部队称的问题,故在今后的工作中,我们还需要分析产生不对称问题的原因,并寻找相应的解决方案。

\end{itemize}

另外,在实验部分只验证了2D场景下的可行性与适用性。本文提出通过增加第三维流体速度场数据的采样与训练,建立对应的过完备稀疏字典,然后在流体速度场的重构上采样步骤中重构流体速度场的三个分量,实现本文提出的过完备稀疏字典技术向3D流体模拟场景的扩展。但是,在具体的实现中,存在以下一些问题和待完成的工作:

\begin{itemize}

\item 提高流体局部细微结构的大小时,稀疏训练字典的列向量维度增长太快。基于稀疏编码的过完备稀疏训练字典技术假设输入的高精度信号是一个列向量,为了使流体单个方向上的速度场数据适用于过完备稀疏字典方法,本文采用的解决方案是将2D或者3D的速度场数据扁平化处理,然后存储到一个列向量中。故假设使用的流体局部细微结构的大小为$n$,那么在2D场景下,稀疏训练字典的每一列的维度为$n^2$,而在3D场景下,稀疏训练字典的每一列的维度则能达到$n^3$。故在3D场景中,不能使用过大的流体局部细微结构。另外,列向量信号维度的快速增长,也给训练稀疏字典时高精度流体速度场训练样本数据的采集带来了困难。

\item 在本文提出的降采样联合训练字典方法中,需要一个降采样矩阵辅助计算适用于流体低精度速度场数据的过完备稀疏字典。但是要设计这样的一个降采样矩阵,需要将扁平化的高精度训练字典还原到3D空间,设计一个降采样假想的3D高精度稀疏字典的3D降采样矩阵,再将其映射成2D矩阵的形式。

\item 重构上采样3D低精度流体速度场数据时,需要逐一处理3D的流体局部细微结构。为了使该局部细微结构可以应用过完备稀疏字典方法重构上采样到高精度速度场空间,需要在应用过完备稀疏字典重构方法前,对其做扁平化处理到一个列向量中;在应用重构上采样方法结束后,再将其还原到3D空间中。

\end{itemize}

根据上述描述,我们可以总结将本文的流体计算框架扩展到3D场景主要需要注意的问题为两点,即大矩阵数据的处理和不同维度数据之间的转换。在后续的工作中,我们会设计更详细的方案解决上述提出的问题。
